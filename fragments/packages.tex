%!TEX root = ../thesis.tex
\usepackage{selinput}
\usepackage[babel, german=quotes]{csquotes}
\usepackage[authoryear]{natbib}
\usepackage{url}
\usepackage[absolute]{textpos}
\usepackage{makecell}
\usepackage{epigraph}
\usepackage{marginnote}
\usepackage{setspace}
\usepackage{xcolor}
\usepackage{dirtree}
\usepackage{listings}
\usepackage{inconsolata}
\usepackage{tabularx}
\usepackage{multirow}
\usepackage[hidelinks]{hyperref}
\usepackage[capposition=top]{floatrow}
\usepackage{afterpage}

\usepackage{graphicx}
\usepackage{enumitem}
\usepackage{epstopdf}
\SelectInputMappings{
  adieresis={ä},
  germandbls={ß},
}
\usepackage[ngerman]{babel}
\usepackage[printonlyused]{acronym}

\setlength{\parskip}\bigskipamount
\setlength{\parindent}{0pt}

\bibliographystyle{natdin}

\lstset{aboveskip=\bigskipamount,belowskip=\medskipamount}

\clubpenalty = 10000 % Schusterjunge
\widowpenalty = 10000 % Hurenkind

% Configs for dictum used in chapterpreamble
\setkomafont{dictumtext}{\itshape}
\setkomafont{dictumauthor}{\normalfont}
\renewcommand*\dictumwidth{\linewidth}
\renewcommand*\dictumauthorformat[1]{\vspace{0.5em}--- #1\vspace{1em}}
\renewcommand*\dictumrule{}

\newcommand{\fullref}[1]{\hyperref[{#1}]{Kapitel \textit{{\ref*{#1} – \nameref{#1}}}}}
\newcommand{\lstref}[1]{\hyperref[{#1}]{Listing \ref*{#1}}}
\newcommand{\figref}[1]{\hyperref[{#1}]{Abbildung \ref*{#1}}}
\newcommand{\tabref}[1]{\hyperref[{#1}]{Tabelle \ref*{#1}}}

\newcommand{\draft}[1]{\color{red}#1}

\newcommand\blankpage{%
    \null
    \thispagestyle{empty}%
    \addtocounter{page}{-1}%
    \newpage}

\definecolor{listingnormal}{HTML}{24292E}
\definecolor{listingred}{HTML}{D73A49}
\definecolor{listingsilver}{HTML}{BABBBD}
\definecolor{listinggrey}{HTML}{6A737D}
\definecolor{listingdarkblue}{HTML}{032F62}
\definecolor{listingblue}{HTML}{005CC5}
\definecolor{listingpurple}{HTML}{6F42C1}

\lstdefinelanguage{elixir}{
  morekeywords = {case,catch,def,do,else,false,%
		use,alias,receive,timeout,defmacro,defp,%
		for,if,import,defmodule,defprotocol,%
		nil,defmacrop,defoverridable,defimpl,%
		super,fn,raise,true,try,end,with,%
		unless},
  alsoletter={\\,=,|,>},
	morekeywords = [2]{=,|>,\\\\},
  morekeywords = [3]{@moduledoc,@doc},
	sensitive=true,
	morecomment=[l]{\#},
  morestring=[b]",
  morestring=[b]',
	morecomment=[s]{@doc\ \"""}{"""},
  keywordstyle=\color{listingred},
  keywordstyle=[2]\color{listingred},
  keywordstyle=[3]\color{listinggrey},
}

\lstdefinelanguage{js}{
  morekeywords = {typeof,new,true,false,catch,function,return,null,catch,%
    switch,var,if,in,while,do,else,case,break,class,export,import,%
    extends,const,let,=,=>},
  morekeywords = [2]{Component,constructor,componentWillMount,%
    componentDidMount,render,connect,default,componenWillUnmount%
    },
  morekeywords = [3]{this,mapStateToProps,mapDispatchToProps},
  keywordstyle=\color{listingred},
  keywordstyle=[2]\color{listingpurple},
  keywordstyle=[3]\color{listingblue},
  sensitive=true,
  comment=[l]{//},
  morecomment=[s]{/*}{*/},
  commentstyle=\color{listinggrey},
  stringstyle=\color{listingdarkblue},
  morestring=[b]',
  morestring=[b]`,
  morestring=[b]"
}


\lstset{ %
  backgroundcolor=\color{white},   % choose the background color; you must add \usepackage{color} or \usepackage{xcolor}; should come as last argument
  basicstyle=\ttfamily\scriptsize\color{listingnormal},        % the size of the fonts that are used for the code
  breakatwhitespace=false,         % sets if automatic breaks should only happen at whitespace
  breaklines=true,                 % sets automatic line breaking
  captionpos=b,                    % sets the caption-position to bottom
  commentstyle=\color{listinggrey},    % comment style
  escapeinside={\%(}{)\%},          % if you want to add LaTeX within your code
  extendedchars=true,              % lets you use non-ASCII characters; for 8-bits encodings only, does not work with UTF-8
  frame=shadowbox,	                   % adds a frame around the code
  keepspaces=true,                 % keeps spaces in text, useful for keeping indentation of code (possibly needs columns=flexible)
         % keyword style
  numbers=left,                    % where to put the line-numbers; possible values are (none, left, right)
  numbersep=10pt,                   % how far the line-numbers are from the code
  numberstyle=\ttfamily\color{listingsilver}, % the style that is used for the line-numbers
  rulecolor=\color{black},         % if not set, the frame-color may be changed on line-breaks within not-black text (e.g. comments (green here))
  showspaces=false,                % show spaces everywhere adding particular underscores; it overrides 'showstringspaces'
  showstringspaces=false,          % underline spaces within strings only
  showtabs=false,                  % show tabs within strings adding particular underscores
  stepnumber=1,                    % the step between two line-numbers. If it's 1, each line will be numbered
  stringstyle=\color{listingdarkblue},     % string literal style
  tabsize=2,	                   % sets default tabsize to 2 spaces
  title=\lstname                   % show the filename of files included with \lstinputlisting; also try caption instead of title
}

% Patch to show correct line numbering starting with firstline
% https://tex.stackexchange.com/questions/26828/first-line-number-in-lstinputlisting-environment
\makeatletter
\patchcmd{\lst@GLI@}% <command>
  {\def\lst@firstline{#1\relax}}% <search>
  {\def\lst@firstline{#1\relax}\def\lst@firstnumber{#1\relax}}% <replace>
  {\typeout{listings firstnumber=firstline}}% <success>
  {\typeout{listings firstnumber not set}}% <failure>
\makeatother
