%!TEX root = ../thesis.tex
\chapter{Einleitung}

\section{Was versteht man unter Deployment?}

Unter \emph{Deployment} (zu deutsch etwa ``Inbetriebsetzung'') versteht man in der Software-Entwicklung alle notwendigen Schritte, um ein System verwendbar zu machen (vgl. \citep[21]{Breuer2009}). Ein häufiger Anwendungsfall in der Praxis ist der Transfer und die Ausführung einer Applikation auf eine Zielumgebung, bspw. einen Produktions-Server.

Manuelle Deployments, die Schritt für Schritt von einer oder gar mehreren Personen durchgeführt werden, sind aus menschlicher Natur fehleranfällig, zeitaufwändig und das Ergebnis oftmals unvorhersehbar – vor allem bei großen und komplexen Applikationen, deren Deployment in vielen Schritten durchgeführt werden muss. \citep[5]{Humble2010} Stattdessen ist es empfehlenswert, den Deployment-Vorgang komplett zu automatisieren. Die Rolle, die dabei der Mensch spielt, beschränkt \citet[][5f]{Humble2010} auf wenige simple Aufgaben:

\begin{quotation}
  There should be two tasks for a human being to perform to deploy software into a development, test or production environment: to pick the version and environment and to press the ``deploy'' button.
\end{quotation}

Die Umsetzung dieses Vorgangs ist die \emph{Deployment Pipeline}. Dabei wird der komplette Deployment-Vorgang in kleinere Schritte aufgeteilt, die sequenziell und teilweise auch parallel ausgeführt werden; beispielsweise beginnend mit dem Kompilieren der Anwendung, über automatisierte Tests bis hin zum Transfer auf die Zielumgebung. \citep{FowlerDP}
