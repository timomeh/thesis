%!TEX root = ../thesis.tex
\setchapterpreamble[ur][.8\textwidth]{%
\dictum[Neal Ford, \textit{The Productive Programmer}]{%
``Computers are designed to do simple repetitive tasks. As soon as you have humans doing repetitive tasks on behalf of computers, they all get together late at night and laugh at you.''}
}

\chapter{Einleitung}

Automatisierung ist nicht nur in der Informatik sondern auch in vielen anderen Branchen erstrebenswert. Eins der bekanntesten historischen Vorbilder ist Sakichi Toyoda, der schon in den 1920er Jahren das Potential der Automatisierung erkannte und einen automatisierten Webstuhl erfand, der Qualität und Effizienz erhöhte und gleichzeitig Produktionskosten verringerte. Später wird er die Toyota Industries Corporation gründen, die bis heute dafür bekannt ist, stets ihre Prozesse zu verbessern.

In der Informatik gibt es viele Stellen, an denen Automatisierung eingesetzt werden kann. Ziel ist es immer, repetitive Prozesse von einem Computer durchführen zu lassen, welcher den Prozess schneller und zuverlässiger durchführen kann.

Diese Arbeit beschäftigt sich mit der Automatisierung von Software Deployments: der Prozess der Inbetriebsetzung einer Software, angefangen von der lokalen Arbeitsstation des Entwicklers bis hin zur Produktionsumgebung und ihrer Nutzer.


\section{Problemstellung}


\section{Motivation}


\section{Ziel der Arbeit}


\section{Aufbau und Vorgehen}
