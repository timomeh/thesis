%!TEX root = ../thesis.tex
\chapter{Fazit}

Um ein Fazit über die Arbeit zu ziehen, werfen wir nochmals einen Blick auf das Ziel, welches in \fullref{ch:einleitung} und im Titel der Arbeit bestimmt wurde: die Programmierung einer eigenen Webanwendung zur Durchführung von Deployments.

Zu Beginn war schon klar, dass es sich nur um eine prototypische Umsetzung handeln kann. Die Entwicklung einer produktionsfähigen Anwendung nimmt weitaus mehr Zeit in Anspruch als zur Bearbeitung verfügbar war. Dazu kommt, dass der Umfang des Themas \emph{Deployment} etwas unterschätzt wurde. Es gibt noch viele Aspekte, die nicht berücksichtigt werden konnten, jedoch tiefergehend betrachtet werden müssen, um Mehrwert und Alleinstellungsmerkmale auszuarbeiten. \emph{Release Management} ist einer dieser Aspekte. Die Integration von Informationen zu Releases kann einen weiteren Vorteil der Anwendung bieten.

Dennoch wurde das Ziel erreicht und ein verteiltes System entwickelt, welches einen Deployment-Prozess durchführt, darstellt, und sogar aufgrund der visuellen Gestaltung und Funktionalität auf sehr positives Feedback gestoßen ist. Eine Weiterentwicklung ist sinnvoll und vielversprechend.

Die Arbeit hat nicht nur den Grundstein für eine potentielle Produktentwicklung gelegt, sondern auch viel Wissen in verschiedensten Bereichen abgedeckt, was in der Praxis einen sehr hohen Wert hat. Man bedenke alleine, dass zu Beginn der Arbeit keinerlei Kenntnisse in Elixir vorhanden waren. Durch die Implementierung des Webservers in Phoenix, und vor allem durch die Entwicklung der Worker mit ihrer Kommunikation untereinander, konnte Elixir Schritt für Schritt in einem Zeitraum von wenigen Wochen gelernt werden. Die Lernkurve von Elixir war sehr angenehm, und letztendlich wurde damit eine Anwendung im hochperformanten Erlang-System entwickelt.

Ebenso konnte viel Erfahrung in der Entwicklung von Single Page Applications mit React und Redux gewonnen werden, obwohl zu Beginn schon einiges an Erfahrung vorhanden war. Die Frontend-Anwendung besitzt eine sehr solide Struktur, die einfach zu verstehen ist und dennoch stark skalieren kann. Auch für die reine Komposition von Komponenten wurde eine vielversprechende Methodik erarbeitet, bei der eine Komponente tatsächlich nur für eine einzige Aufgabe zuständig ist. Dies baut weiter auf dem Wissen auf, was im Praxisprojekt \citep{Maemecke2017} erlangt wurde. Somit lassen sich die erarbeiteten Architekturmuster in vielen anderen Projekten anwenden.

Nicht zu vernachlässigen ist das Wissen über Arbeitsabläufe im Zusammenhang mit Deployments wie Continuous Integration und Continuous Delivery. Durch die korrekte Durchführung dieser Arbeitsabläufe lassen sich effizient stabile Anwendungen im Team entwickeln und der Nutzer erhält schneller neue Features. Davon profitieren Nutzer, Kunde und auch der Entwickler. Zu Continuous Delivery gehören auch viele Themen aus dem Bereich DevOps, die während der Erarbeitung kennengelernt wurden.

Abschließend lässt sich zusammenfassen, dass die Implementierung von WARP zeigt, welche Bereiche die Entwicklung moderner Webanwendungen umfasst, und was ein Fullstack-Entwickler dafür an Erfahrung mitbringen muss. Diese Erfahrungen können in zukünftigen Projekten genutzt werden, weiter ausgebaut werden und auch an andere Entwickler weitergegeben werden.
