%!TEX root = ../thesis.tex
\chapter{Grundlagen automatisierter Deployment-Prozesse}

Vor der Programmierung einer Anwendung sollte man sich mit der Thematik familiär machen, in der sich die Anwendung bewegt. Interessant am Thema \emph{Deployments} im Zusammenhang einer eigenen Anwendung ist, dass sie direkt das nutzen kann, was sie anbietet: Die funktionierende Anwendung übernimmt automatisierte Prozesse für andere Anwendungen; sie kann aber auch selbst ihre eigenen automatisierten Prozesse übernehmen.

Auch das hier erarbeitete Wissen und die Methoden können im Entwicklungsprozess der Anwendung verwendet werden.

\section{Definitionen und Begriffserklärungen}

Vorweg sollen jene Begriffe und deren Zusammenhang erklärt werden, die zum Verständnis der Arbeit notwendig sind; vor allem die Begriffe \emph{Deployment}, \emph{automatisiert} und \emph{Webanwendung} aus dem Titel der Arbeit.

\subsection{Definition: Deployment}

Unter Deployment (zu deutsch etwa ``Inbetriebsetzung'') versteht man in der Soft\-ware-Entwicklung alle notwendigen Schritte, um ein System verwendbar zu machen \citep[21]{Breuer2009}. Dies beinhaltet u.a. das Ausführen von Tests, das kompilieren von ausführbaren Dateien oder anderen Artefakten, und dem Transfer auf eine Zielumgebung \citep[4]{Humble2010}.

Beim Deployment einer Webanwendungen sind jene Schritte beispielsweise:

\begin{itemize}
  \item Ausführen der Tests des Clients und des Servers
  \item Kompilieren von Artefakten wie CSS- und JavaScript-Dateien
  \item Transfer auf einen Server
  \item Ausführen eines Rauchtests
\end{itemize}

\subsection{Definition: Automatisierung}

Der Duden \citeyearpar{Duden} definiert \emph{automatisiert} u.a. als ``durch Selbststeuerung oder -regelung erfolgend''. Spezifischer für Software beschreibt \citet[27]{Duvall2007} einen Vorgang als automatisiert, wenn nach seinem Start kein weiteres menschliches Eingreifen mehr benötigt wird. Demnach handelt es sich bei einem automatisierten Deployment um die Inbetriebsetzung einer Software ohne menschliches Handeln nach initialem Auslöser.

Im Gegensatz dazu steht die manuelle Durchführung eines Prozesses. Im Kontext des Deployments müssen dabei einige oder gar alle Schritte zur Inbetriebsetzung durch menschliches Handeln durchgeführt werden.

Dies bedeutet jedoch nicht, dass ein automatisierter Prozess für den Nutzer gänzlich unsichtbar ist. Um auf Erfolg oder Fehlschlag des Automatismus reagieren zu können, muss der Prozess seinen Status an den Nutzer zurück melden. Vorteilhaft sind auch Informationen während des Prozessablaufs, beispielsweise um dessen Fortschritt zu verfolgen. \citep[10f]{Duvall2007}

\subsection{Definition: Webanwendung}

Eine Webanwendung ist eine Anwendung, die auf einem entfernten Server ausgeführt wird und vom Nutzer mit einen Browser über \ac{HTTP} aufgerufen wird. Durch die Aufteilung auf Server und Browser spricht man daher vom Client-Server-Modell.

Moderne Webanwendungen werden immer öfter als verteilte Systeme programmiert. Hierbei wird der Server über ein \ac{API} angesprochen (zu deutsch \emph{Schnittstelle}), zu welcher der Client dynamisch mittels JavaScript eine grafische Repräsentation ausgibt. Eine solche Serveranwendung wird auch Service genannt. \citep[Kapitel 5.1.3]{Fielding} beschreibt dies als Client-Stateless-Server-Modell.

\subsection{Begriffserklärung: DevOps}
\label{subsec:devops}

Auch wenn der Begriff DevOps nicht viel Verwendung in dieser Arbeit finden wird\footnote{Der Begriff DevOps beschreibt ein weitgreifendes Konzept mit mehreren Aspekten. Daher wird in dieser Arbeit die Verwendung spezifischerer Begriffe bevorzugt.}, so sollte er auch nicht verschwiegen werden, da er stark mit der Thematik in Verbindung steht.

DevOps setzt sich aus den zwei englischsprachigen Wörtern \emph{Development} und \emph{Operations} zusammen. Zu deutsch also die Verbindung von Entwicklung und Systembetrieb. DevOps ist \emph{keine} Berufsgruppe, wie oft fälschlicherweise angenommen wird, sondern eine Kultur, die Entwicklung und Systembetrieb stärker miteinander verbindet.

Im weiteren Sinne ist DevOps die Weiterführung agiler Entwicklungsmethoden auf technischer Basis: Bei der agilen Entwicklung wird die Zusammenarbeit und Flexibilität zwischen Entwicklern und Unternehmen so verbessert, dass auf neue Anforderungen direkt eingegangen werden kann, damit jene schneller veröffentlicht werden (``Business Agility''). DevOps ergibt sich daraus, dass auch die Zusammenarbeit zwischen Entwicklern und IT Administratoren verbessert wird (``IT Agility''). \citep[4f]{Chapman2014}

Demnach ist ein Aspekt von DevOps die Verwendung automatisierter Prozesse zum schnellen und zuverlässigen Deployment von Anwendungen.

\subsection{Definition und Eingrenzung: Build}
\label{subsec:build}

\citet[27]{Duvall2007} definiert den Begriff Build als ``a set of activities performed to generate, test, inspect and deploy software''. \emph{Build} wird hierbei als Prozess definiert, welcher in \fullref{sec:deployment-pipeline} näher beleuchtet wird.

Ein Build kann auch als Ergebnis des Building-Vorgangs interpretiert werden, bei dem eine Software als kompilierte Binärdatei oder – abhängig von der Programmiersprache – auch als Quellcode inklusive generierter Artefakte erstellt wird. Ein solcher Build stellt die fertige Anwendung dar und wird mit einer eindeutigen Versionsnummer versehen.

Um im weiteren Verlauf der Arbeit Verwechselungen beider Interpretationen zu vermeiden, wird der Begriff \emph{Build-Prozess} verwendet.

\subsection{Erklärung und Beispiele: Testing}
\label{subsec:def-testing}

Tests überprüfen die Qualität und Anforderungen einer Anwendung. Tests, die automatisiert durchgeführt werden können, werden bereits vom Entwickler geschrieben und helfen dabei, Fehler frühzeitig zu erkennen und zu beheben. Hierzu gehören Komponententests (engl. unit test), welche kleinste Einheiten der Anwendung überprüfen, und Integrationstests (engl. integration test), die das Zusammenspiel verschiedener Komponenten testen. Ebenso ermöglichen jene Tests das fehlerfreie Umstrukturieren von Code (enlg. refactoring). \citep[104f]{Wolff2016}

Eine weitere Art von Test ist der Akzeptanztest (engl. acceptance test), bei dem die Funktionalität der Anwendung für den Nutzer überprüft wird. Diese können vom Kunden, einer \ac{QA} oder gar dem Nutzer durchgeführt werden. Akzeptanztests können automatisiert werden, werden oft aber auch manuell durchgeführt.

Der Rauchtest (engl. smoke test) ist ein sehr einfach umzusetzender Test, bei dem die Software gestartet wird und danach überprüft wird, ob sie noch läuft oder abgestürzt ist. Bei Webanwendungen kann hierfür die Startseite aufgerufen werden und nach einer Zeichenkette gesucht werden, die dort auftauchen sollte.

Letztlich handelt es sich bei einem Build, wie in \fullref{subsec:build} definiert, indirekt ebenfalls um einen Test. Ein Build kann fehlschlagen, was z.B. auf Syntaxfehler oder Abweichungen in der Systemumgebung hinweisen kann. Ein erfolgreicher Build gibt die Sicherheit, dass in diesen Aspekten keine Fehler vorhanden sind.

{\draft NEEDS MORE CITES?}

\section{Arbeitsabläufe mit automatisierten Deployments}

Sobald mehrere Entwickler an einem Projekt arbeiten, ist ein strukturierter Arbeitsablauf notwendig. Im Bezug auf automatisierte Deployments gibt es drei etablierte Verfahrensweisen, die aufeinander aufbauen: \emph{Continuous Integration}, \emph{Continuous Delivery} und \emph{Continuous Deployment}.

Auch ohne zu wissen, was hinter diesen Begriffen steckt, ist die Gemeinsamkeit \emph{``continuous''} erkennbar; zu Deutsch übersetzbar als \emph{kontinuierlich} oder – noch passender: \emph{ununterbrochen}. Wir können jetzt schon darauf schließen, dass in diesen Verfahrensweisen ständig etwas getan wird.

\subsection{Continuous Integration}

Neue oder geänderte Funktionalitäten einer Software können nur selten innerhalb weniger Stunden entwickelt werden. Manche Features benötigen mehrere Tage, Wochen oder gar Monate zur Fertigstellung. Wenn während\-dessen die Anwendung von anderen Entwicklern noch weiterentwickelt wird, kommt irgendwann der Zeitpunkt, an dem jene Änderungen \emph{integriert} werden müssen.

Nun besteht die Gefahr, dass die Integration der Änderungen in die mittlerweile veränderte Anwendung weiteren Aufwand mit sich bringt. Dieses Szenario soll durch \emph{Continuous Integration} verhindert werden.

\ac{CI} bedeutet die ununterbrochene Integration von Än\-der\-ung\-en in die Anwendung. Ein Entwickler soll dabei mindestens einmal am Tag seine Änderungen in den Hauptentwicklungszweig (Trunk) integrieren. Bei jeder Integration muss die Anwendung stabil bleiben, alle Tests müssen erfolgreich sein und die Anwendung muss buildbar sein. Sollte eine Integration fehlschlagen, muss sie mit höchster Priorität behoben werden.

Das Ergebnis von \ac{CI} ist ein Trunk, der zu jedem Zeitpunkt stabil ist und die neusten Änderungen der Software beinhaltet. Um \ac{CI} erfolgreich betreiben zu können und um den Aufwand für den Entwickler zu reduzieren, sollte das stetige Testen neuer Integrationen von einem Automatismus übernommen werden.

Am Begriff \emph{Trunk} ist zu erkennen, dass diese Definition aus früheren Zeiten der Versionskontrolle mit Subversion (SVN) stammt. Das Arbeiten in Branches über längere Zeit ist mit heutigen Tools wie Git weitaus einfacher. Ein Entwickler muss nicht mehr den Aufwand betreiben und seinen Fortschritt täglich in den Hauptbranch (standardmäßig ``master'' genannt) integrieren.

Das Gegenteil lässt sich nämlich mit gleichem Ergebnis weitaus einfacher umsetzen: Täglich soll der master-Branch in den Branch des Entwicklers integriert werden. Dadurch besteht die Sicherheit, dass der eigene Branch integrierbar bleibt und somit die Änderungen jederzeit problemlos in den master-Branch übernommen werden können.

{\draft NEEDS MORE CITES! CD/CI BOOK!}

\subsection{Continuous Delivery}

Continuous Delivery ist die Weiterführung von Continuous Integration. Da der Hauptbranch jederzeit durch einen Automatismus getestet wird, ist es nach den Tests auch möglich, den Transfer und die Ausführung der Anwendung, inklusive aller damit verbundenen Schritte, auf eine beliebige Zielumgebung automatisiert durchzuführen.

Das bedeutet nicht, dass nach der Integration zwangsläufig ein Deployment erfolgen muss. Es soll nur möglich sein, die Anwendung ohne weiteren Aufwand automatisiert zu deployen. Dadurch kann beispielsweise ein Tester eigenständig eine bestimmte Version der Anwendung auf eine Testumgebung aufspielen, genauso wie auch ein neuer Release auf einem Produktionsserver ohne Aufwand aufgespielt werden kann.

Eine große Rolle spielt hierbei die Konfiguration und das automatische Erstellen der Zielumgebung (vgl. \fullref{subsec:devops}).

Das Konzept des Continuous Delivery ist eng verbunden mit der agilen Arbeitsmethode:

\begin{quote}
  Our highest priority is to satisfy the customer through early and continuous delivery of valuable software.
\end{quote}

Da Continuous Delivery eng mit automatisiertem Deployment in Verbindung steht, werden oft beide Begriffe synonym verwendet.

{\draft NEEDS MORE CITES! CD/CI BOOK!}


\subsection{Continuous Deployment}

Continuous Deployment geht das Konzept von Continuous Delivery noch einen Schritt weiter. Hierbei werden alle integrierten Änderungen direkt auf eine Umgebung deployed.

Eins der bekanntesten Beispiele für Continuous Deployment ist Amazon: Im Mai 2010 wurde bei Amazon durchschnittlich alle 11,6 Sekunden ein Deployment durchgeführt, bei denen 10000 bis maximal 30000 Server gleichzeitig ein Deployment empfangen haben.\footnote{http://assets.en.oreilly.com/1/event/60/Velocity\%20Culture\%20Presentation.pdf} Im November 2014 hat Amazon bekanntgegeben, dass sie innerhalb eines Jahres 50 Millionen Deployments\footnote{Zur Relation: Ein Jahr hat ungefähr 31,5 Millionen Sekunden. Es wurde also durchschnittlich mehr als ein Deployment pro Sekunde durchgeführt.} auf verschiedene Umgebungen durchgeführt haben.\footnote{http://www.allthingsdistributed.com/2014/11/apollo-amazon-deployment-engine.html}

\section{Die Deployment-Pipeline als Prozess-Manifestation}
\label{sec:deployment-pipeline}

\citet[106]{Humble2010} erklärt den Begriff \emph{Deployment-Pipeline} in wenigen Sätzen:

\begin{quote}
 At an abstract level, a deployment pipeline is an automated manifestation of your process for getting software from version control into the hands of your users. Every change to your software goes through a complex process on its way to being released. That process involves building the software, followed by the progress of these builds through multiple stages of testing and deployment.
\end{quote}

Eine Pipeline (zu deutsch etwa ``Rohrleitung'') kann auch bildlich als eine Rohrleitung vorgestellt werden. Sie besteht aus mehreren Segmenten, die miteinander verbunden werden. Anstatt dass eine Flüssigkeit durch diese Rohrleitung fließt, wird der Quelltext der Anwendung in diese Rohrleitung, durch ihre Segmente gereicht.

\subsection{Aufbau einer Deployment-Pipeline}

Eine Deployment-Pipeline beschreibt einen Prozess, in dem der Quelltext Schritt für Schritt zu einer lauffähigen Anwendung kompiliert wird, bis sie letztendlich durch mehrere \emph{Stages} auf die Produktionsumgebung transferiert wird. Die Pipeline besteht aus solchen Stages, die nacheinander ausgeführt werden. In einer Stage werden zusammengehörige Aufgaben parallel oder seriell ausgeführt. Die Stage muss erfolgreich abgeschlossen werden, damit die nächste Stage ausgeführt werden kann.

Ein Beispiel für eine Stage ist die \emph{Commit Stage}. In ihr werden Tests ausgeführt und die Software kompiliert. Ergebnis dieser Stage ist eine Anwendung, die auf technischer Basis lauffähig ist. \citep[110]{Humble2010}

Eine technisch funktionierende Anwendung ist jedoch noch nicht zwangsläufig eine Anwendung, die auch direkt an den Nutzer gegeben werden kann. Demnach kann als zweite Stage beispielsweise ein Akzeptanztest (siehe \fullref{subsec:def-testing}) durchgeführt werden.

Somit werden nacheinander Stages aufgerufen, bis die Anwendung zuletzt ver\-öf\-fent\-licht werden kann.

Einer der wichtigsten Aspekte für Continuous Integration ist das Integrieren und Testen der Anwendung, was der ersten Stage entspricht. Zur Integration müssen keine weiteren Stages ausgeführt werden. Es können also unterschiedlich viele Stages in einer Pipeline ausgeführt werden, je nachdem, ob weitere Stages überhaupt notwendig sind.

\subsection{Anforderungen an eine Deployment-Pipeline}

Deployment-Pipelines müssen immer idempotent sein: egal wie oft man eine Pipeline mit dem gleichen Quelltext aufruft, sie muss immer das gleiche Ergebnis haben. Daher sollte eine Pipeline immer auf dem gleichen System ausgeführt werden; bestenfalls eine Maschine, welche alle Pipelines ausführt. \citep[155]{Humble2010}

Eine weitere Anforderung ist das zeitliche Steuern der Pipeline. Als krassestes Beispiel dürfen zwei Pipelines nicht simultan ein Deployment auf die gleiche Umgebung durchführen. Dies könnte zu unvorhersehbaren Ergebnissen führen. Stages, die sich gegenseitig beeinflussen können, müssen demnach nacheinander ausgeführt werden. \citep[119]{Humble2010}

In der oben genannten Definition wird die Pipeline nach jeder Änderung ausgeführt. Es gibt auch Abläufe, bei denen eine Pipeline in regelmäßigen Intervallen ausgeführt wird, z.B. jede Nacht (``Nightly Build''). Bei Webanwendungen ist dies meist jedoch nicht der Fall.

\subsection{Anforderungen an eine Deployment-Anwendung}

\citet[127]{Humble2010} nennt als wichtige Anforderung bei der Implementierung eines eigenen Systems eine \emph{Deployment Page}. Dabei handelt es sich um eine Auflistung, welche Softwareversion auf welche Umgebung deployed wurde. Diese \emph{Deployment Page} kann auch andere Softwareversionen anzeigen, die noch nicht deployed wurden, aber deployed werden können.

Nun nähern wir uns immer mehr dem Thema Release Management, welches verwandt mit Deployment-Prozessen ist. Zur Eingrenzung der Arbeit wird Release Management allerdings nicht weiter behandelt.
