\documentclass{scrartcl}
\usepackage{selinput}
\usepackage[babel, german=quotes]{csquotes}
\usepackage[authoryear]{natbib}
\usepackage{url}

\usepackage{graphicx}
\usepackage{epstopdf}
\SelectInputMappings{
  adieresis={ä},
  germandbls={ß},
}
\usepackage[ngerman]{babel}

\setlength{\parindent}{0pt}
\setlength{\parskip}\bigskipamount

\begin{document}

\titlehead{\includegraphics[height=30pt]{assets/logo_th-koeln.eps}}
\subject{Exposé zur Bachelor Thesis}
\title{Entwurf und Implementierung einer Webapplikation zur grafisch gestützten Durchführung von automatisierten Deployments}
\author{Timo Mämecke}
\publishers{Erstprüfer: Prof. Christian Noss\\
Zweitprüfer: Dirk Breuer
}
\maketitle

\tableofcontents

\section{Problemstellung}
\label{sec:problemstellung}

\subsection{Was versteht man unter Deployment?}

Unter \emph{Deployment} (zu deutsch etwa ``Inbetriebsetzung'') versteht man in der Software-Entwicklung alle notwendigen Schritte, um ein System verwendbar zu machen (vgl. \citep[21]{Breuer2009}). Ein häufiger Anwendungsfall in der Praxis ist der Transfer und die Ausführung einer Applikation auf eine Zielumgebung, bspw. einen Produktions-Server.

Manuelle Deployments, die Schritt für Schritt von einer oder gar mehreren Personen durchgeführt werden, sind aus menschlicher Natur fehleranfällig, zeitaufwändig und das Ergebnis oftmals unvorhersehbar – vor allem bei großen und komplexen Applikationen, deren Deployment in vielen Schritten durchgeführt werden muss. \citep[5]{Humble2010} Stattdessen ist es empfehlenswert, den Deployment-Vorgang komplett zu automatisieren. Die Rolle, die dabei der Mensch spielt, beschränkt \citet[][5f]{Humble2010} auf wenige simple Aufgaben:

\begin{quotation}
  There should be two tasks for a human being to perform to deploy software into a development, test or production environment: to pick the version and environment and to press the ``deploy'' button.
\end{quotation}

Die Umsetzung dieses Vorgangs ist die \emph{Deployment Pipeline}. Dabei wird das komplette Deployment in kleinere Schritte aufgeteilt und nacheinander ausgeführt, beispielsweise beginnend mit dem Kompilieren der Anwendung, über automatisierte Tests bis hin zum Transfer auf die Zielumgebung. \citep{FowlerDP}

\subsection{Automatisiertes Deployment in der Praxis}

Obwohl eine Deployment Pipeline der bewährteste Prozess für schnelle und zuverlässige Deployments ist, wird sie – nach eigener Erfahrung, durch die Arbeit mit verschiedenen Unternehmen und Projekten – oftmals nicht verwendet; auch wenn die Nachteile des manuellen Deployments für die Beteiligten deutlich spürbar sind. Gründe dafür sind:

\begin{itemize}
\item Das Aufsetzen der Pipeline wird als sehr kompliziert wahrgenommen.
\item Das Aufsetzen und die Wartung der Pipeline nimmt Zeit in Anspruch.
\item Fehlendes Wissen und Erfahrung im Themengebiet.
\item Vorhandene applikatorische Lösungen\footnote{zu den bekanntesten zählen u.a. Jenkins, Capistrano und Atlassian Bamboo} sind tatsächlich sehr umfangreich, komplex und verlangen tiefgehendes Wissen.
\end{itemize}

Letztendlich wird die Thematik dadurch nicht nur für Firmen schwer greifbar, sondern auch für Einsteiger und Fortgeschrittene, die davon profitieren können.

\section{Fragestellung}
\label{sec:fragestellung}
Lorem ipsum dolor sit amet, consectetur adipisicing elit, sed do eiusmod tempor incididunt ut labore et dolore magna aliqua.

\section{Ziel}
\label{sec:ziel}
Lorem ipsum dolor sit amet, consectetur adipisicing elit, sed do eiusmod tempor incididunt ut labore et dolore magna aliqua. Ut enim ad minim veniam, quis nostrud exercitation ullamco laboris nisi ut aliquip ex ea commodo consequat. Duis aute irure dolor in reprehenderit in voluptate velit esse cillum dolore eu fugiat nulla pariatur. Excepteur sint occaecat cupidatat non proident, sunt in culpa qui officia deserunt mollit anim id est laborum.

\section{Aufbau der Arbeit}
\label{sec:aufbau}
Lorem ipsum dolor sit amet, consectetur adipisicing elit, sed do eiusmod tempor incididunt ut labore et dolore magna aliqua. Ut enim ad minim veniam, quis nostrud exercitation ullamco laboris nisi ut aliquip ex ea commodo consequat. Duis aute irure dolor in reprehenderit in voluptate velit esse cillum dolore eu fugiat nulla pariatur. Excepteur sint occaecat cupidatat non proident, sunt in culpa qui officia deserunt mollit anim id est laborum.


\bibliographystyle{natdin}
\bibliography{references}

\end{document}
