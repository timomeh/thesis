\documentclass{scrartcl}
\usepackage{selinput}
\usepackage[babel, german=quotes]{csquotes}
\usepackage[authoryear]{natbib}
\usepackage{url}

\usepackage{graphicx}
\usepackage{enumitem}
\usepackage{epstopdf}
\SelectInputMappings{
  adieresis={ä},
  germandbls={ß},
}
\usepackage[ngerman]{babel}

\newlist{tocpreview}{enumerate}{2}
\setlist[tocpreview]{label*=\arabic*.}

\setlength{\parskip}\bigskipamount

\begin{document}

\titlehead{\includegraphics[height=30pt]{./assets/logo_th-koeln.eps}}
\subject{Exposé zur Bachelor Thesis}
\title{Entwurf und Implementierung einer Webapplikation zur grafisch gestützten Durchführung von automatisierten Deployments}
\author{Timo Mämecke\\Matrikelnr.: 11088767}
\date{14. August 2017}
\publishers{
  \vspace{\bigskipamount}
  \large{
    \begin{tabular}{ l l }
      Erstprüfer: & Prof. Dipl. Des. Christian Noss \\
      Zweitprüfer: & Dirk Breuer
    \end{tabular}
  }
}
\maketitle


\section{Problemstellung}
\label{sec:problemstellung}

\subsection{Was versteht man unter Deployment?}
\label{sec:deployment-what}

Unter \emph{Deployment} (zu deutsch etwa ``Inbetriebsetzung'') versteht man in der Software-Entwicklung alle notwendigen Schritte, um ein System verwendbar zu machen (vgl. \citep[21]{Breuer2009}). Ein häufiger Anwendungsfall in der Praxis ist der Transfer und die Ausführung einer Applikation auf eine Zielumgebung, bspw. einen Produktions-Server.

Manuelle Deployments, die Schritt für Schritt von einer oder gar mehreren Personen durchgeführt werden, sind aus menschlicher Natur fehleranfällig, zeitaufwändig und das Ergebnis oftmals unvorhersehbar – vor allem bei großen und komplexen Applikationen, deren Deployment in vielen Schritten durchgeführt werden muss. \citep[5]{Humble2010} Stattdessen ist es empfehlenswert, den Deployment-Vorgang komplett zu automatisieren. Die Rolle, die dabei der Mensch spielt, beschränkt \citet[][5f]{Humble2010} auf wenige simple Aufgaben:

\begin{quotation}
  There should be two tasks for a human being to perform to deploy software into a development, test or production environment: to pick the version and environment and to press the ``deploy'' button.
\end{quotation}

Die Umsetzung dieses Vorgangs ist die \emph{Deployment Pipeline}. Dabei wird der komplette Deployment-Vorgang in kleinere Schritte aufgeteilt, die sequenziell und teilweise auch parallel ausgeführt werden; beispielsweise beginnend mit dem Kompilieren der Anwendung, über automatisierte Tests bis hin zum Transfer auf die Zielumgebung. \citep{FowlerDP}

\subsection{Automatisiertes Deployment in der Praxis}
\label{sec:deployment-real}

Nach eigener Erfahrung – durch die Arbeit an verschiedenen Projekten und Unternehmen – werden automatisierte Deployments in der Praxis oftmals nicht oder nur teilweise verwendet. Das Einrichten der Deployment Pipeline ist komplex und verlangt Expertise, die unter Umständen im Unternehmen nicht vorhanden ist. Es nimmt Zeit in Anspruch, womit unmittelbar auch Kosten verbunden sind. Die Anschaffung eines automatisierten Deployments ist somit eine Investition, welche keinen direkten Vorteil des eigentlichen Produkts mit sich bringt: die Applikation wird dadurch nicht schneller, und natürlich werden dadurch keine Fehler behoben oder gar neue Features eingeführt; alles Aufgaben, die eine höhere Priorität haben.

Stattdessen ermöglicht das automatisierte Deployment ein zuverlässigeres und schnelleres Veröffentlichen von Änderungen an der Software. Der Release-Zyklus kann deutlich verkürzt werden, und fehlerhafte Features schneller zurückgenommen werden. \citep[11f]{Humble2010} All diese Merkmale sind auf längere Sicht sehr vorteilhaft und erstrebenswert, aber nicht direkt ersichtlich, wenn man nicht mit der Thematik vertraut ist.

Oft sieht man bei der Entwicklung von Webanwendungen auch, dass programmatisch generiert Dateien\footnote{z.B. bei der Konkatenation und Minifizierung von JavaScript oder CSS, oder auch beim Kompilieren von Stylesheet-Sprachen wie Sass, Less oder Stylus} durch den Entwickler lokal generiert und dem Versionskontrollsystem (VCS) hinzugefügt werden, wenn keine Deployment Pipeline vorhanden ist. Dies hat zumindest den Vorteil, dass Änderungen manuell mit wenigen Befehlen von einem Entwickler aus dem VCS auf die Zielumgebung aufgespielt werden können, da sich die notwendigen generierten Artefakte bereits im VCS befinden. Kompilierte Dateien sollten allerdings nicht dem VCS hinzugefügt werden: dadurch entstehen unnötige Merge-Konflikte, und letztendlich handelt es sich um redundante Daten. Wie in Sektion \ref{sec:deployment-what} beschrieben, sollte der Entwickler diese Schritte nicht manuell durchführen.

Die Verwendung von Versionskontrollsystemen ist weit verbreitet, demnach ist der notwendige Schritt zur automatisierten Deployment Pipeline ein Server, der mit dem VCS-Repository verbunden ist und daraus Build-Prozesse ausführt, aber auch Tests zentralisiert durchführt und bei Erfolg die Applikation auf die Zielumgebung transferiert.

Applikatorische Lösungen für solche Deployment Server existieren zwar, wirken aber weniger attraktiv:

\begin{itemize}
  \item Viele User Interfaces entsprechen nicht mehr den heutigen Erwartungen.
  \item Bekannte Systeme sind komplex, da sie eine Vielzahl von Applikationsumgebungen abdecken.\footnote{bspw. nicht nur Webapplikationen, sondern auch Smartphone-Apps}
  \item Jenkins\footnote{https://jenkins.io/ – ``The leading open source automation server'', auf Java basiert}, eine der bekanntesten Lösungen, muss auf einem Server installiert werden. Es fallen Kosten für den Server und Aufwand aufgrund der Installation an.
\end{itemize}

Letztendlich wird die Thematik dadurch nicht nur für Unternehmen schwer greifbar, sondern auch für Einzelpersonen, die davon profitieren können.

\section{Motivation}

Die in Sektion \ref{sec:deployment-real} beschriebenen Rückstände sind mir bei meiner Arbeit als Web Entwickler in letzten Jahren oft begegnet. Selbst große Unternehmen, deren Hauptgeschäft über E-Commerce Webapplikationen abgewickelt wird, hatten Probleme eine Möglichkeit zum komplett automatisierten Deployment, passend zu den eigenen Anforderungen und Kompetenzen der Angestellten zu finden.

Für dieses Problem ist ein Markt vorhanden, der nicht nur von einer Applikation zum automatisierten Deployment profitieren kann; sie bietet auch einen vereinfachten Einstieg in die Thematik und einen Ausgangspunkt zur (selbstständigen) Weiterbildung des Nutzers.

Letztendlich ermöglicht mir diese Bachelor Thesis, selbst in das Thema Deployment und Dev\-Ops einzutauchen: Erfahrung und Expertise darin hat große Relevanz und Nutzen in der Praxis.

\section{Ziel}
\label{sec:ziel}

Ziel der Bachelor Thesis ist die Entwicklung einer prototypischen Webapplikation als verteilte Client-Server-Anwendung zur Durchführung von automatisierten Deployments von Webapplikationen. Die Einschränkung auf Webapplikationen als Gegenstand des Deployments grenzt bewusst den Umfang und die Komplexität des Themas auf die zur Verfügung stehende Bearbeitungszeit ein. Die programmierte Applikation soll keine produktionstaugliche Webapplikation werden, sondern zunächst eine prototypische Umsetzung mit dem Ausblick, sie weiter ausbauen zu können.

Der Server soll dabei die Deployment-Pipeline implementieren, eine Schnittstelle zur Kontrolle der Deployments anbieten und den Status des Deployments in Echtzeit melden. Der Clients soll als Browser Frontend dem Benutzer eine einfache Oberfläche nach dem Grundsatz in Sektion \ref{sec:deployment-what} zur Verfügung stellen: Auswahl der Version und Umgebung, und klick auf einen ``deploy''-Button.

Weiterhin soll der Client die gesamte Pipeline und den aktuellen Status des Deployments ansprechend visualisieren. In Anlehnung an das abgeschlossene Praxisprojekt soll der Client komponentenbasiert umgesetzt werden, um die theoretischen Ergebnisse des abgeschlossenen Praxisprojekts praktisch anzuwenden.

Zur kontinuierlichen Evaluation der Lösungsansätze wird die Anwendung mit sich selbst deployed.

\section{Vorgehen}
\label{sec:vorgehen}

Zunächst soll die Bachelorarbeit theoretisch ausarbeiten, was man unter Deployment versteht und wie es eingesetzt wird. Es gibt verschiedene Dienste und Applikationen, die eine Deployment Pipeline implementieren, demnach bietet es sich an, die Eigenschaften dieser Lösungen in einer Marktanalyse auszuarbeiten. In Anbetracht der zur Verfügung stehenden Bearbeitungszeit soll die Analyse nur verkürzt behandelt werden, da der Hauptbestandteil der Arbeit auf der Implementation einer eigenen Applikation liegt. Vor der eigentlichen Implementation wird die Webapplikation aus technischer und visueller Seite entworfen. Die Implementation teilt sich letztendlich, wie üblich für Client-Server-Anwendungen, in die Implementation des Servers und die Implementation des Clients auf. Dabei wird auch auf die Entwicklung in einer Virtualisierung wie Docker\footnote{https://www.docker.com} oder Vagrant\footnote{https://www.vagrantup.com} eingegangen: einerseits als Entwicklungsumgebung, andererseits werden Virtualisierungen auch bei der Implementierung von Deployment-Services verwendet\footnote{vgl. bspw. Bitbucket Pipelines (https://blog.bitbucket.org/2017/04/18/bitbucket-pipelines-supports-building-docker-images-and-service-containers/)}.

\pagebreak

\section{Vorläufige Gliederung}
\label{sec:gliederung}

\begin{tocpreview}
  \item Einleitung
    \begin{tocpreview}
      \item Problemstellung
      \item Ziel
    \end{tocpreview}
  \item Was ist Deployment?
    \begin{tocpreview}
      \item Deployment Pipeline
      \item Continuous Integration
      \item Continuous Delivery
      \item Continuous Deployment
    \end{tocpreview}
  \item Marktanalyse
  \item Entwurf der Webapplikation
    \begin{tocpreview}
      \item Entwurf der Systemarchitektur
      \item Entwurf eines User Interfaces
    \end{tocpreview}
  \item Implementation der Webapplikation
    \begin{tocpreview}
      \item Implementation des Servers
      \item Implementation des Clients
    \end{tocpreview}
  \item Zusammenfassung der Ergebnisse
  \item Ausblick
\end{tocpreview}

\pagebreak


\bibliographystyle{natdin}
\bibliography{references}

\end{document}
